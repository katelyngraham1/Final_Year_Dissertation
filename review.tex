\chapter{Technology Review}
One of the most critical aspects of software development is choosing the right technology stack. This chapter will discuss the technology stack used for the development of this mobile application. Jira was used to track and manage the project's progress from start to finish. The main project technologies used to undergo this project can be split into two categories: frontend and backend. The former consists of the React Native application. Expo was used to simplify the testing and deployment for the frontend of this project. The latter consists of a Node.js Rest API with Express, Sequelize and MySQL. Postman was used to test the backend before working on the front end.


\section{Jira}
Jira is a project management tool developed by Atlassian\cite{atlassian}. It is designed to help teams plan, track and manage projects efficiently. It is becoming more common for teams in the software development industry to use an agile development methodology, with many teams adopting the use of sprints to organise their work. Sprint planning helps break down complex projects into smaller, more manageable tasks, allowing teams to focus on specific goals and deliverables. This enables better project planning and more accurate estimation of the time and resources required to complete the project.  Sprints promote collaboration and communication between team members. It can help avoid isolation or miscommunication and ensures that everyone is on the same page, leading to better results and higher quality output. Users can create and assign user stories to team members, set deadlines and track progress through the sprint. Additionally, Jira provides a range of project management tools, including time tracking, reporting and custom workflows, to help teams stay organised and focused.
\newline \newline
At first, Jira’s interface can be a bit overwhelming for a new user, but once familiarised with its layout, it is relatively easy to use. The platform is highly customizable, allowing the user to tailor the interface to their specific needs. However, the customization options can also make the platform somewhat complex and time consuming to set up. Jira integrates with a range of other tools commonly used in software development, e.g. Git. This integration allows teams to easily share information and collaborate across different platforms, making it easier to manage complex projects. 
Jira is a popular asset due to its scalability. It can be used by teams of all sizes, from as small as an individual project to large enterprises. 

\section{React Native}

React Native is a popular open-source mobile application development framework. \cite{danielsson2016react}It enables developers to build native mobile apps using JavaScript and React, a popular web development framework. React Native is built on top of React, which allows developers to use a declarative programming model to describe the desired user interface of an application. This approach can make it easier to build and maintain code and can also improve performance. 
\newline \newline
One of the key benefits of React Native is that it enables developers to build mobile apps for both IOS and Android platforms. \cite{RNbenefit} This can save time and effort compared to building separate codebases for each platform and can also make it easier to maintain and update code. Due to React Native using a declarative programming model, it allows developers to describe the desired user interface, rather than how to achieve it. This approach can make it easier to develop and maintain code, as well as improve the performance of the application. 
\newline \newline
React Native also offers several other benefits. Cross platform compatibility is one of the core features of React Native, which allows developers to write a single codebase that can run on both IOS and Android devices. This is achieved by using a set of core components and APIs that abstract away the platform specific differences and allow the app to run consistently across both platforms. Some of the key features and benefits of cross platform compatibility are:
\begin{itemize}
    \item Single codebase: With React Native, developers can write a single codebase that can be used to build apps for both IOS and Android devices, which can save time and effort compared to developing separate apps for each platform.
    \item Native performance: Despite the use of a single codebase, React Native apps are still able to deliver native performance and user experience on both platforms. This is achieved by using platform specific code where necessary, such as handling UI rendering and interactions
    \item Faster development cycle: React Native can also help to speed up the development cycle by allowing for faster iterations and testing. Since changes made to the codebase are immediately reflected on both platforms, developers can quickly test and iterate on their app’s features and user experience.
\end{itemize}	
Another key benefit of React Native is its large and active developer community. Ways in which a large developer community can be beneficial for React Native are as follows: 
\begin{itemize}
    \item Access to support and resources: With a large developer community, there are many resources available for developers to learn and troubleshoot issues with React Native. This includes online forums, blogs, tutorials and documentation which can help speed up the development process and provide solutions to common problems.
    \item Open-source contributions: React Native is an open-source project. This means that developers  can contribute to the development of the framework and its associated tools. With a large developer community, there are many opportunities for developers to contribute code, bug fixes and new features, which can help to improve the overall quality and functionality of React Native.
    \item Career opportunities: Finally, a large developer community can provide career opportunities for developers who specialise in React Native. With a growing demand for mobile app development, particularly for cross-platform apps, developers with React Native expertise are in high demand, and can benefit from a strong and supportive community.
\end{itemize}
Overall, the benefits of a large developer community in React Native are many and varied. With a large and active community, developers of the framework can leverage third-party libraries and tools, drive innovation and growth and benefit from career opportunities in the mobile app development industry.
\newline \newline
React Native provides a rich set of core components and APIs for developers to build mobile apps. However, there may be cases where developers need to add additional functionality or extend the capabilities of their app beyond what is provided in the core framework. In these cases, developers can turn to third-party modules or libraries that have been developed by the community. Third-party modules in React Native are typically published as NPM packages, which can be installed using the npm or yarn package manager. Once installed, these modules can be imported into React Native app like any other JavaScript module. There are many third-party modules available for React Native covering a wide range of use cases, from UI components and animations to data storage and networking. Some popular third-party modules for React Native include:
\begin{itemize}
    \item React Navigation: This is a library that provides a flexible and extensible way to handle navigation and routing in React Native apps.
    \item React Native Elements: A UI toolkit that provides a set of customizable and easy to use UI component.
    \item Axios: A library that provides a simple and powerful way to make HTTP requests in React Native apps.
\end{itemize}
While third-party modules in React Native can provide valuable additional functionality, there are some limitations and challenges to consider when working with them.
\newline \newline
 It is important to consider the compatibility of the module with the version of React Native being used. Third-party modules may not always be compatible with the latest version of React Native or other modules that are already installed in your project. This can lead to issues such as crashes or unexpected behaviour. Some third-party modules may impact the performance of your app, either by increasing memory usage or slowing down the rendering of UI components. It’s important to carefully review the documentation, evaluate the performance impact and compatibility information for each module before installing it in your project and consider alternatives if necessary.
 \newline \newline
React Native has other limitations \cite{RNlimitation} such as access to native functionality. React Native provides access to many native device features, there may be some functionality that is not available, which can limit the scope of certain applications. An example of the limited access to native functionality can include the advanced camera features. React Native provides basic camera functionality but does not support more advanced features such as manual focus or manual exposure control. Although React Native provides a wide range of built in UI components, there may be some cases where developers need to create custom UI components that are not available in the framework. React Native also does not support all types of app extensions, such as keyboard extensions or SiriKit extensions on IOS.
\newline \newline
Another limitation with React Native is the learning curve that is associated with the framework. There are many similarities between both React and React Native as React Native is based on the React library web development, but although there are many similarities, there are some important differences between the two. A developer that is familiar with React may have to learn new concepts such as the differences between components and screen, as well as the structure of a typical React Native app, when starting to program with React Native. React Native has a rapid pace of development and is constantly introducing regular updates to the framework, which often brings new features and improvements, but these updates can also introduce problems such as breaking code or making developers update their existing code. This can then add additional time and effort for the developer to be able to stay up to date with the latest version of React Native and make sure there is compatibility with other third-party modules or libraries.
\newline \newline
In conclusion, React Native is a powerful and versatile mobile development framework that can enable developers to build native mobile apps with a single codebase. Its cross-platform compatibility, large developer community and other benefits make it an attractive option for building mobile apps. However its limitations with respect to access to third-party module compatibility, native functionality and learning curve should also be considered when evaluating its suitability for a particular project. Despite these limitations, many developers find that the benefits of React Native, outweigh the limitations. By investing time and effort in learning React Native and staying up to date with the latest updates and best practices, developers can effectively leverage the framework to build high quality mobile apps.
       
\section{Expo Go}

Expo Go is an open-source platform that allows developers to build, test and deploy native mobile applications using JavaScript and React Native. \cite{expoGo} Expo Go is popular with developers that want to make cross-platform apps that work with IOS, Android and the web.
\newline \newline
One of Expo Go’s main advantages is that it offers a set of tools that reduce the development process, this makes it both faster and more efficient. Hot reloading is a significant benefit when using Expo Go as it allows the developers to see the code changes instantly without having to restart and wait for the application to load up. Due to this function it can be helpful when testing and debugging apps. This often means the developer spends less time fixing problems, improving the overall performance of the app.
\newline \newline
Expo Go gives access to a variety of pre-built components and libraries that are easy to add into the application. By using this, developers are able to reduce the time in which it takes to create complicated applications with innovative features. Additionally, Expo Go offers a number of features that may be used to improve the functionality of the app. This includes push notifications, in app purchases and much more. Due to the services being fully integrated with the Expo platform, these services make it simple to update the apps functionality as needed.
\newline \newline
Expo Go is widely regarded as being a dependable and stable platform with decent support for both IOS and Android smartphones in terms of performance. Expo provides excellent support for new technologies such as augmented reality (AR) and virtual reality (VR). These new technologies can be combined with an app using the Expo AR and Expo VR libraries. 
\newline \newline
Another benefit of Expo Go is that it supports a shorter development process. This can be especially helpful for small teams and individual developers who might not have access to the tools and support that a bigger development team would have. This can lower expenses and increase productivity, enabling developers to concentrate on making high quality programs that satisfy their consumers expectations. 
\newline \newline
Although Expo Go is a reliable software, it does contain some limitations that developers should be aware of before starting to use it when undergoing a project. It does not offer the same level of customization and control as other platforms, which could be a disadvantage to more experienced developers who would need more freedom and control over their code. The pre-built components and libraries may not be suitable for some developer’s particular needs which may then prolong the development process and cost them time. A strong understanding and knowledge of JavaScript and React Native are needed when using Expo Go. If a developer is unfamiliar with these technologies, they could end up needing additional training or assistance in order to use the platform efficiently. 
\newline \newline
Overall, Expo Go is a strong and flexible platform that offers a variety of tools and services to assist developers in producing high quality mobile applications. \cite{expoWithRN} Similarly to React Native, it contains limitations that can make it hard for developers to use but the advantages weigh out the disadvantages as it is widely regarded as a dependable and effective option for programmers who want to build cross-platform applications that work on iOS, Android and the web.



\section{Node.js}
Node.js is an open-source, cross-platform runtime environment for JavaScript that helps developers to create scalable and high-performance applications. \cite{node} It is built on top of the Google V8 JavaScript engine and provides an event-driven, non-blocking I/O model, making it perfect for developing real-time applications. Node.js can make an effective backend tool for applications, especially when used alongside the Express web framework, Sequelize and MySQL database.
\newline \newline
Using Node.js has many advantages, \cite{nodePro&Cons} one of them being its ability to handle multiple concurrent requests. This is due to the server’s non-blocking I/O model, which allows it to handle many requests simultaneously without causing any other requests to be blocked. Node.js can effectively handle high traffic volumes and as a result of this there is very little lagging. Another benefit to using Node.js is that it has a large and active community that is always developing new modules and libraries, making it simpler for developers to create complex applications. Additionally, there are a lot of third-party packages for Node.js that are accessible through npm. This can speed up the development by offering pre-built functionality that can be used in most projects.
\newline \newline
It is easy to create a scalable and effective REST API that can handle multiple requests at the same time using Node.js. The Express framework offers a set of tools and features that make it simple to create routes middleware and handle errors, making it very well-suited to developing RESTful APIs. It is easy to manage data and relationships between tables when using Sequelize to work in an object-oriented way with a SQL database. With regards to a particular stack, using Node.js with Express, Sequelize and MySQL, it provides a solid and effective basis for building a backend. MySQL is a popular and reliable relational database. It offers quick and scalable data storage. Sequelize makes the process of working with MySQL a lot simpler. It provides an object-relational mapping layer which allows the developer to work with the database in an object-oriented way.
\newline \newline
Like most technologies Node.js has its limitations. Some of the limitations in which Node.js has is there is no built-in support for relational databases. Although it can work with MySQL using ORM libraries such as Sequelize, it doesn’t have any built-in support for when a developer is working directly with SQL databases. This means developers have no choice but to rely on third-party libraries and tools to work with SQL databases. There can also be issues with backward compatibility. As Node.js has a short enough release cycle, newer versions may not be able to work with older code.  
\newline \newline
Overall Node.js is a powerful tool when it comes to building a backend that can handle high traffic loads efficiently. It offers a scalable and reliable framework for building a RESTful API when used with Express, Sequelize and MySQL. Although it has some minor limitations, it has a large and active community that is always developing new modules and libraries, which makes it a lot easier for developers to build complex applications.



\section{Sequelize}
Sequelize is an Object-Relational Mapping (ORM) library for Node.js. \cite{sequel} It offers an easy way to work with relational databases like MySQL, SQLite and many more. It provides excellent support for database synchronisation, eager loading, association, transactions and database migration. This can help simplify the database management process. With Sequelize, developers can easily synchronise their database schema with their application models, load similar data in a single query, define complex relationships between models and manage database migration.
Sequelize also includes security features that can help prevent SQL injection attacks. This is a common enough attack for web applications. It provides built-in parameterised queries, making it more difficult for attackers to inject malicious code into SQL queries. Along with that, it works well with other popular Node.js libraries and frameworks, such as Express. This means that developers can easily integrate Sequelize into their already existing applications and use its powerful features without having to completely re-design their application.
However, Sequelize may not be the best choice for every project. First of all, it has a steep learning curve, especially for developers that are new to ORMs and/or are not familiar with JavaScript Promises. Also, while it provides a convenient way to work with databases, not all are suitable as they could be too complex or need highly customisable database interaction. In these cases, it may be necessary to write raw SQL queries or use a different ORM.
Overall it is a powerful ORM and has many features that can really simplify the process of working with relational databases in a Node.js and Express environment. 



\section{MySQL}
MySQL is an open-source relational database management system. \cite{sql} It is consistently ranked as the most popular database for developers, according to surveys from websites such as Stack Overflow and JetBrains. The main reasons it is so popular with developers is due to its high performance, reliability and ease of use, and due to the fact that it is Open Source (free to use). MySQL supports a variety of popular development languages including Go, Ruby, Rust, C, C++, Python, etc, as well as previously mentioned Node.js/ JavaScript.
\newline \newline
The MySQL Database is known as a client/server system that includes a multithreaded SQL server that supports several backends, many client programs and libraries, as well as administrative tools and a wide range of applications-programming interfaces (APIs). MySQL is also provided as an embedded multithreaded library that can be linked into an application to get a smaller, faster, easier to manage individual product.
\newline \newline
MySQL is fast, reliable, scalable and easy to use. It was initially developed to be able to quickly handle large databases and has since been used in extremely demanding production applications. Despite it being under constant development, MySQL offers a set of functions that are rich and useful. It’s connectivity, speed and security make it highly suited for accessing databases on the internet. 
\newline \newline
There are an endless number of key benefits \cite{sqlAd&Disad} for using MySQL. It is easy to use as Developers can install MySQL in minutes and the database is easy to manage. It is also extremely reliable, in fact it is one of the most mature and widely used databases. As it has been around for over twenty-five years, it has been tested on a wide variety of scenarios and by many of the world’s largest companies. Due to its great reliability many of these big organisations depend on MySQL to run critical business applications. The scalability makes MySQL very appealing as even organisations like Facebook are able to scale their application to support billions of users. 
\newline \newline
MySQL Heat Wave is known to be less expensive and quicker than other database services. This was demonstrated by multiple standard industry benchmarks, including TPC-H and CH-benCHmark. For high availability and disaster recovery, MySQL offers a complete set of native, fully integrated replication technologies. Users can get maximum flexibility from the MySQL Document Store when developing SQL and NoSQL database applications. Developers are able to mix and match relational data and JSON documents in the same database and application. 
\newline \newline
Although MySQL is a highly popular database system one of its flaws is that it has limited support for unstructured data. Even though MySQL can store binary data, it lacks native support for unstructured data types like photo images and videos. Because of this, it is not as suitable for applications that need to store a lot of unstructured data.



\section{Postman}
Postman is a popular collaboration platform for API development that provides tools that can be used for testing, documenting and sharing APIs.\cite{PostmanPlatform} Developers and testers use it frequently to speed up the development of APIs to make sure they function as intended.
\newline \newline
Postman is well known for being easy to use. It provides a user-friendly interface that allows developers to quickly and easily create and test API requests. It offers a large range of tools for working with JSON data and supports a wide variety of HTTP methods. It also provides a powerful testing framework that allows developers to automate API testing, this can help ensure that APIs are working correctly and can save developers time and effort when developing applications. GitHub and Slack are just two of the many other tools and services that Postman offers integration with. By using these tools it could improve the API development workflow’s streamlining and make it simpler to include testing and other operations. \cite{IntroPostman}
\newline \newline
Postman is a great platform for testing individually, but it also provides collaboration features that allows developers to use APIs and work with others on API development. It allows developers to track changes and roll back changes, if necessary, with its version control feature. Lastly, it provides tools for creating API documentation that can easily be shared with other developers. Using this can help ensure that APIs are well-documented and easy to understand.
\newline \newline
Like every technology, Postman has its limitations. One of the limitations with Postman is that it has limited functionality when used in complex scenarios. While it is a popular, powerful tool when it comes to testing APIs, it may not always be suitable to use with bigger applications that require specialised testing tools. Postman is designed to work with HTTP based APIs but is not overly suitable for testing APIs that use other protocols like TCP or UDP. Also, it stores API credentials and other sensitive information in its cloud server. This could be a security concern for certain organisations.
