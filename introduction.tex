\chapter{Introduction}	

In today’s digital age, file management and organisation are critical aspects of productivity and efficiency. However, traditional paper-based filing systems are no longer sufficient to meet the demands in the modern world. Due to the ever-rising development of technology, there is a need for a more user-friendly and efficient way of organising digital invoices for small businesses.  This is where the filing system web application comes in. The title of this project is “File A While”. The objective of this project is to design and develop a web-based application that facilitates effortless storage of invoices for the end users by offering a user-friendly interface. 

\section{Overview} 

This chapter aims to provide the reader with a comprehensive understanding of the project’s background and its primary objectives. Chapter 2 will describe the methodology that was taken into consideration when approaching the project, including the various collaborative tools utilised by the developer. In Chapter 3 a conceptual level overview of the primary technologies used in this project will be presented. This sets the foundation for Chapter 4, which will be an in-depth discussion of the system design. Chapter 5 will discuss the system evaluation and reflect on the objectives that were outlined in the introduction. It will also address the system’s current limitations and potential for future development. The sequence of the chapters taken together will provide a full analysis of the project’s scope and implementation.

\section{Background and Scope} 

The filing system is an essential tool for individuals and organisations to store and manage information effectively. With the increasing reliance on digital storage, there is a growing need for efficient and user-friendly filing systems that can adapt to changing needs and take advantage of the new technologies available. In the context of this project, the present study aims to develop a filing system web application that utilises both frontend and backend technologies. The application in relation to this dissertation will be developed using React Native for the front end and Node.js for the backend. This dissertation will provide a comprehensive overview of the development process. It will highlight the challenges, solutions and outcomes of the project. By addressing key research questions related to the development process. 
\newline \newline
The development of web applications has seen rapid growth in recent years, with businesses and organisations turning to these applications to engage with customers and manage their operations. This trend has been driven by the increasing availability and accessibility of mobile devices and internet connectivity. 
\newline \newline
With the growing demand for web applications, there is a need for developers to create applications that are user-friendly, scalable and secure. The development of a Filing System Management Application requires a deep understanding of the principles of frontend and backend development, as well as the ability to design a system that is intuitive and meets the needs of users. In this context, the present study aims to develop a filing system web application that leverages the strengths of React Native and Express.js/Node.js, providing users with an efficient and user-friendly platform to manage their information.
\newline \newline
The traditional methods of filing and managing information, such as physical paper-based systems, are becoming outdated with the rise of digitalization. However, while digital filing systems offer several advantages over their physical counterparts, they can be complicated and difficult to manage, particularly when dealing with large amounts of data. Existing filing systems may be slow, outdated or insecure, creating inefficiencies and security risks for organisations and individuals alike. Additionally, many existing filing systems do not provide a user-friendly interface, making it difficult for users to access and manage their information. In response to these challenges, the present study aims to develop a filing system web application that addresses the shortcomings of existing systems, providing users with a platform that is efficient, user-friendly and secure.
\newline \newline
The purpose of this study is to develop a filing system web application that utilises React Native for the frontend and Express.js / Node.js for the backend. The application will be designed to provide users with an intuitive and efficient platform to manage their information. The study will explore the various design considerations and challenges associated with developing such an application, including user interface design, database management, security and scalability. The study aims to answer several research questions related to the development, such as: What are the key design considerations for developing a user-friendly and efficient filing system web application? What are the best practices for integrating frontend and backend technologies to create a cohesive application? How can security risks be mitigated when developing a web application that deals with sensitive information? By answering these questions, the study aims to contribute to the growing body of knowledge on effective web application development and provide practical insights for developers seeking to create user-friendly and secure filing system applications.

\section{Objectives} 

With the background provided above, the below will outline the main objectives set for this project and personal objectives set out by the developer.
\subsection{Project}
\begin{itemize}
    \item To develop a mobile application which enables small businesses to efficiently process their invoice management by providing a convenient digital platform to organise and file invoices with ease.
    \item Develop an application that will make the management of invoices easier for small enterprises by providing a digital platform to organise and store their invoices. 
    \item Eliminate the need for excessive paper documentation and simplify the process of tracking invoices.
    \item Develop a fully operational web application with minimal defects 
\end{itemize}

\subsection{Personal}
    \begin{itemize}
        \item Learn how to use React Native effectively and feel confident applying it
        \item Increase knowledge of JavaScript programming language
        \item Create a minimally flawed application that is fully functioning and user-friendly
        \item Create an application that helps small businesses to manage and store their invoices online, eliminating the need for endless paperwork
    \end{itemize}

\section{Work Breakdown}
This section will give an overview of the work carried out in this project

\subsection{Project} 
\begin{enumerate}
    \item Research conducted throughout the project
    \item Planning of the applications using wire frames, Jira, etc.
    \item Backend
        \begin{itemize}
            \item Create a basic Node.js App
            \item Set up web server
            \item Configure MySQL database and Sequelize
            \item Define the Sequelize Model
            \item Create the controller (CRUD objects)
            \item Define Routes
            \item Test the API
        \end{itemize}
    \item Frontend
        \begin{itemize}
            \item Initialise an Expo app
            \item Splash Screen
            \item Drawer Navigation
            \item Log In Screen
            \item Registration Screen
            \item Home Screen
            \item Setting Screen
            \item Screen to View All Invoices
            \item Add Invoice Screen
            \item Individual Invoice Summary Screen
        \end{itemize}
    \item Integrate and Test
    \item GitHub
\end{enumerate}
\subsection{Dissertation}
\begin{enumerate}
    \item Research
    \item Abstract
    \item Introduction
    \item Methodology
    \item Technology Review
    \item System Design
    \item System Evaluation
    \item Conclusion
\end{enumerate}

\section{Source Code}
This section contains the link to the relevant GitHub repository for this project. 

\subsection{Applied Project}

The below URL directs to the primary Git repository used for this project’s development. It contains all the code from the frontend and backend.

\href{https://github.com/katelyngraham1/Final_Year_Project}{https://github.com/katelyngraham1/Final\textunderscore Year\textunderscore Project} 


\subsection{Dissertation} 
This URL points to the source control for this latex document. Readers can refer to the above repository for more detail on the project’s implementation.

\href{https://github.com/katelyngraham1/Final_Year_Dissertation}
{https://github.com/katelyngraham1/Final\textunderscore Year\textunderscore Dissertation}



\subsection{Video Demo}
This URL points to the demo video. This video goes through the application and explains how to use the it.

\href{https://youtu.be/oJY_FlK4dSM}
{https://youtu.be/oJY\textunderscore FlK4dSM}



