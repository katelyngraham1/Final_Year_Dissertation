\chapter{Conclusion} 

The main objective of the current project was to design and build an application that would make it simple for end users to store and manage invoices. The goal of the developer was to create a simple interface for users that would do away with the requirement for extensive paper documentation and simplify the tracking of invoices. The project also provided the developer with the chance to strengthen their programming skills in JavaScript and React Native, which would be extremely helpful in the future when developing new applications. 
\newline \newline
In hindsight, the project has, to a considerable part, succeeded in achieving its goals. The created application efficiently streamlines the handling and storage of invoices while offering users a user-friendly substitute for the challenging paper-based invoice tracking system. The user interface has been thoughtfully created with a focus on usability and effective navigation, eventually improving the user experience. Additionally, the developer has effectively gained proficiency in JavaScript/ NodeJS and React Native development, which can be used in future projects.
\newline \newline
The chapter on system evaluation provided insightful information about the planning, testing, and general application design. The developer initially had trouble following the sprint approach during the planning stage, which caused problems with task management and progress monitoring. The developer decided to carefully adhere to the sprint technique following the holiday break, which resulted in a more efficient and well-organised development process. The two-week sprint intervals turned out to be the best option because they balanced the effort and allowed for enough time to make significant growth progress.
\newline \newline
In terms of testing, the developer used Postman to carefully review the application's backend functioning. This approach made it possible to find problems early on and fix them, guaranteeing the consistency and dependability of the backend code. As a result, the application's overall performance improved, which benefited the end users. An efficient analysis of test results was made possible by Postman's user-friendly interface, which made it a useful tool for backend testing.
\newline \newline
A close examination of the application's design revealed that the mechanism for filing invoices generally functioned well and was deemed appropriate for small businesses needing a solution for monitoring the payment and invoicing processes. The developer admitted that the folder and company backend components' full potential had not yet been achieved. As a result, the developer advised adding a dropdown list with alternatives for company names to the name input box when producing a new invoice. The process of invoicing would be simplified, and efficiency would improve. The developer also proposed making a panel for companies so users could examine all firms and pick them to view the bills that go with them. With this improvement, managing invoicing and payments for various businesses would be easier to manage, both in terms of organisation and usability.
\newline \newline
In summary, the project has been successful in achieving its core objectives of developing an invoice management application that is both user-friendly and efficient. The application provides users with an easy-to-use solution and tackles the main problems associated with traditional paper-based invoice tracking solutions. 
\newline \newline
In conclusion, the created application provides a strong base for further growth and development. The programme can develop into a more complete and reliable solution for invoice management by adding new features, improving user experience, and improving the performance and security. There is tremendous potential for the programme to develop into an essential tool for companies and individuals looking for a cutting-edge and effective method of managing their invoices as long as the developer keeps building upon what has been achieved and lessons learnt from this project.
